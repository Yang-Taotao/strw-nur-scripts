% ==================================================================================== %
\documentclass{article}
\usepackage[a4paper]{geometry}
\usepackage[english]{babel}
\usepackage{parskip}
\usepackage{amsmath, amssymb}
\usepackage{graphicx}
\usepackage{subcaption}
\usepackage{float}
\usepackage[hidelinks]{hyperref}
\usepackage{listings}
\usepackage{xcolor}
% ==================================================================================== %
\definecolor{codegreen}{rgb}{0,0.6,0}
\definecolor{codegray}{rgb}{0.5,0.5,0.5}
\definecolor{codepurple}{rgb}{0.58,0,0.82}
\definecolor{backcolour}{rgb}{0.95,0.95,0.92}
% ==================================================================================== %
\lstdefinestyle{mystyle}{
  backgroundcolor=\color{backcolour},
  commentstyle=\color{codegreen},
  keywordstyle=\color{magenta},
  numberstyle=\tiny\color{codegray},
  stringstyle=\color{codepurple},
  basicstyle=\ttfamily\footnotesize,
  breaklines=true,
  captionpos=b,
  keepspaces=true,
  numbers=left,
  numbersep=5pt,
  showstringspaces=false,
  tabsize=2
}
\lstset{style=mystyle}
% ==================================================================================== %
\title{NUR A - Assignment 1}
\author{Taotao Yang (s4866835)}
\date{Dated: \today}
% ==================================================================================== %
\begin{document}
% ==================================================================================== %
  \maketitle

  % \textit{
  %   Write a short description of your solution in each section. 
  %   Explain your choices. Draw the conclusions from your results.
  % }

  % \textit{
  %   You can refer to selected lines of code where you think it is necessary. 
  %   However, you must provide the full code below each question/task.
  % }

  % \textit{
  %   Don't forget to describe your figures 
  %   and mention what conclusions you draw from them!
  % }
% ==================================================================================== %
  \section{Poisson Distribution}

    We start with the general expression for Poisson distribution:
    \begin{equation}
      P_{\lambda}(k) = \frac{\lambda^{k} e^{-\lambda}}{k!}
    \end{equation}

    To prevent overflow/underflow in general, we circumvent factorial calculation via
    mapping to log space. We therefore rewrite Poisson distribution as:
    \begin{equation}
      \begin{aligned}
        \ln{\left(P_{\lambda}(k)\right)} 
        &= \ln{\left(\frac{\lambda^{k} e^{-\lambda}}{k!}\right)} \\
        &= k\ln{\left(\lambda\right)} - \lambda - \ln{k!} \\
        &= k\ln{\left(\lambda\right)} - \lambda - \sum_{i=1}^{k+1}\ln\left(i\right)
      \end{aligned}
    \end{equation}

    Therefore, the actual distribution is recovered via:
    \begin{equation}
      P_{\lambda}(k) = \exp{\left(\ln{\left(P_{\lambda}(k)\right)}\right)}
    \end{equation}

    To ensure input parameter $\lambda$ and $k$ follows the correct \texttt{dtype},
    we add error handling to function via simple if-else statements. 
    We enforce correct \texttt{dtype} via local \texttt{dtype} conversion in function.
    Additionally, for normalized Poisson distribution, we set special case where $k=0$. 
    That is, Poisson distribution simplify to:
    \begin{equation}
      P_{\lambda}(k) = \frac{\lambda^{k} e^{-\lambda}}{k!} 
      = \frac{\lambda^{0} e^{-\lambda}}{0!} = e^{-\lambda} ~;~ k = 0
    \end{equation}

    These steps further simplifies the calculations. That is, we can recover the actual
    value of Possion distribution from the log space results via a simple exponential.
    This is on top of converting a factorial into a summation of terms.
    
    % Add the full code below.
    The full code used for this question:
    \lstinputlisting[language=Python]{../../output/a1q1_poisson_code.txt}

    % Show your results.
    The results of $P_{\lambda}(k)$ for selected values of $k$ and $\lambda$ 
    are shown in Table~\ref{tab:poisson}.

    \begin{table}[H]
      \centering
      \caption{Poisson probability distribution for selected $k$ and $\lambda$.}
      \label{tab:poisson}
      \begin{tabular}{|c|c|c|}
        \hline
        $\lambda$ & $k$ & $P_{\lambda}(k)$ \\ 
        \hline
        \input{../../output/a1q1_poisson_output.txt}
      \end{tabular}
    \end{table}
% ==================================================================================== %
  \section{Vandermonde Matrix and Interpolation}

    \subsection{(2a) LU decomposition}
      \begin{figure}[H]
        \centering
        \includegraphics[width=0.80\textwidth]{../../plots/a1q2_vandermonde_sol_2a.pdf}
        \caption{
          Polynomial fit evaluated using LU decomposition. 
          Top: data points and interpolated curve. 
          Bottom: absolute error at the data points on a log scale.
        }
        \label{fig:vandermonde_2a}
      \end{figure}

    \subsection{(2b) Neville's algorithm}
      \begin{figure}[H]
        \centering
        \includegraphics[width=0.80\textwidth]{../../plots/a1q2_vandermonde_sol_2b.pdf}
        \caption{
          Interpolation using Neville's algorithm. 
          Top: data points and interpolated curve. 
          Bottom: absolute error at the data points on a log scale.
        }
        \label{fig:vandermonde_2b}
      \end{figure}

    \subsection{(2c) Improving the LU decomposition}
      \begin{figure}[H]
        \centering
        \includegraphics[width=0.80\textwidth]{../../plots/a1q2_vandermonde_sol_2c.pdf}
        \caption{
          LU-based solution with iterative refinement (showing iterations 0, 1 and 10). 
          Top: interpolated curves. 
          Bottom: absolute error at the data points on a log scale.
        }
        \label{fig:vandermonde_2c}
      \end{figure}

    \subsection{(2d) Timing}

      Timing results (average per run):
      \begin{itemize}
        \input{../../output/a1q2_execution_times.txt}
      \end{itemize}

    \subsection{Code for Question 2}
      The following code was used for parts (2a)--(2d):
      \lstinputlisting[language=Python]{../../output/a1q2_vandermonde_all_code.txt}

    \subsection{Conclusions}
      \textit{
        Write a short summary here: 
        which method is most accurate, 
        where errors increase, 
        and how iterations affect accuracy.
      }
% ==================================================================================== %
  \section*{Acknowledgement}
      This article is rendered off of source files at 
      \href{https://github.com/Yang-Taotao/strw-nur-scripts}{strw-nur-scripts}.
% ==================================================================================== %
\end{document}
% ==================================================================================== %
